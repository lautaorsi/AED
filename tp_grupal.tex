\documentclass[10pt,a4paper]{article}

\input{AEDmacros}
\usepackage{caratula} % Version modificada para usar las macros de algo1 de ~> https://github.com/bcardiff/dc-tex


\titulo{Trabajo Practico}

\fecha{\today}

\materia{Algoritmos y Estructuras de Datos}
\grupo{Grupo Algo Tango}

\integrante{Orsi, Lautaro Manuel}{689/23}{Lautaorsi@gmail.com}
\integrante{Apellido, Nombre2}{002/01}{email2@dominio.com}
\integrante{Apellido, Nombre3}{003/01}{email3@dominio.com}
\integrante{Apellido, Nombre4}{004/01}{email4@dominio.com}
% Pongan cuantos integrantes quieran

% Declaramos donde van a estar las figuras
% No es obligatorio, pero suele ser comodo
\graphicspath{{../static/}}

\begin{document}

\maketitle

\section{Especificaci\'on}
\subsection{Redistribucion De Los Frutos}

\begin{proc}{nombre}{\In recursos : \TLista{\ent}, \In cooperan: \TLista{\bool}}{\TLista{\ent}}
	\requiere{|recursos| == |cooperan|}
	\asegura{(\forall i : \ent)(0\le i  < \longitud{recursos} \land  \Biggl(\biggl((cooperan[i] == True) \implies res[i] == \dfrac{\sum\limits_{i=0}^{|recursos|} recursos[j]}{|recursos|}\biggr) \lor \biggl((cooperan[i] == False) \implies res[i] == recursos[i] + \dfrac{\sum\limits_{i=0}^{|recursos|} recursos[j]}{|recursos|})\biggr)\Biggr)}
\end{proc}







\subsection{Trayectoria De Los Frutos Individuales a Largo Plazo}

\begin{proc}{nombre}{\Inout trayectorias: \TLista{\TLista{\float}}, \In cooperan: \TLista{\bool}, \In apuestas:
\TLista{\TLista{\float}}, \In pagos: \TLista{\TLista{\float}}, \In eventos: \TLista{\TLista{\nat}}}{}
	\requiere{expresionBooleana1}
	\asegura{expresionBooleana2}
	\aux{auxiliar1}{parametros}{tipoRes}{expresion}
\end{proc}



\subsection{Trayectoria Extraña Escalera}

\begin{proc}{trayectoriaExtrañaEscalera}{\In trayectoria: \TLista{\float}}{\bool}
	\requiere{|trayectoria| > 0}
	\asegura{(\forall u : \ent) \Biggl(  0 \leq i < |trayectoria| \land \biggl( \Bigl(  (1 \leq |trayectoria| \leq 2) \implies (res == True) \Bigr) \lor \Bigl(  hayUnicoMax(trayectoria) \implies  (res == True)\Bigr)   \biggr) \Biggr)}
	\pred{hayUnicoMax}{\In list: \TLista{\float}}{} 
\end{proc}

\subsection{Individuo Decide Si Cooperar O No}

\begin{proc}{individuoDecideSiCooperarONo}{\In individuo: \nat, \In recursos: \TLista{\float}, \Inout cooperan: \TLista{\bool}, \In apuestas: \TLista{\TLista{\float}}, \In pagos: \TLista{\float}, \In eventos: \TLista{\TLista{\nat}}}{}
	\requiere{expresionBooleana1}
	\asegura{expresionBooleana2}
	\aux{auxiliar1}{parametros}{tipoRes}{expresion}
	\pred{pred1}{parametros}{expresion} 
\end{proc}


\subsection{Individuo Actualiza Apuesta}

\begin{proc}{individuoActualizaApuesta}{\In individuo: \nat, \In recursos: \TLista{\float}, \In cooperan: \TLista{\bool}, \Inout apuestas: \TLista{\TLista{\float}}, \In pagos: \TLista{\TLista{\float}}, \In eventos: \TLista{\TLista{\nat}}}{}
	\requiere{expresionBooleana1}
	\asegura{expresionBooleana2}
	\aux{auxiliar1}{parametros}{tipoRes}{expresion}
	\pred{pred1}{parametros}{expresion} 
\end{proc}







\section{Demostracion de correctitud}

\begin{proc}{nombre}{\In paramIn : \nat, \Inout paramInout : \TLista{\ent}}{tipoRes}
	%    \modifica{parametro1, parametro2,..}
	\requiere{expresionBooleana1}
	\asegura{expresionBooleana2}
	\aux{auxiliar1}{parametros}{tipoRes}{expresion}
	\pred{pred1}{parametros}{expresion} 
\end{proc}

\aux{auxiliarSuelto}{parametros}{tipoRes}{expresion}
% \paraTodo{variable}{tipo}{expresion}
% \existe{variable}{tipo}{expresion}
% Pueden tener [unalinea] para que no se divida en varias lineas
\pred{predSuelto}{parametros}{\paraTodo[unalinea]{variable}{tipo}{algo \implicaLuego expresion}}
\pred{predSuelto}{parametros}{\existe[unalinea]{variable}{tipo}{algo \yLuego expresion}}



\end{document}
