\documentclass[10pt,a4paper]{article}

\input{AEDmacros}
\usepackage{caratula} % Version modificada para usar las macros de algo1 de ~> https://github.com/bcardiff/dc-tex


\titulo{Trabajo Practico}

\fecha{\today}

\materia{Algoritmos y Estructuras de Datos}
\grupo{Grupo AlgoTango}

\integrante{Orsi, Lautaro Manuel}{689/23}{Lautaorsi@gmail.com}
\integrante{Zerbetto De Palma, Gerardo Gabriel}{900/22}{g.zerbetto@gmail.com}
\integrante{Simoza Sanchez, Valeria Andreina}{1027/22}{vsimoza.vs@gmail.com}
\integrante{Apellido, Nombre4}{004/01}{email4@dominio.com}
% Pongan cuantos integrantes quieran

% Declaramos donde van a estar las figuras
% No es obligatorio, pero suele ser comodo
\graphicspath{{../static/}}

\begin{document}

\maketitle

\section{Especificaci\'on}
\subsection{Redistribucion De Los Frutos}

\begin{proc}{nombre}{\In recursos : \TLista{\ent}, \In cooperan: \TLista{\bool}}{\TLista{\ent}}
	\requiere{|recursos| == |cooperan|}
	\asegura{(\forall i : \ent)(0\le i  < \longitud{recursos} \land  \Biggl(\biggl((cooperan[i] == True) \implies res[i] == \dfrac{\sum\limits_{i=0}^{|recursos|} recursos[j]}{|recursos|}\biggr) \lor \biggl((cooperan[i] == False) \implies res[i] == recursos[i] + \dfrac{\sum\limits_{i=0}^{|recursos|} recursos[j]}{|recursos|})\biggr)\Biggr)}
\end{proc}







\subsection{Trayectoria De Los Frutos Individuales a Largo Plazo}

\begin{proc}{nombre}{\Inout trayectorias: \TLista{\TLista{\float}}, \In cooperan: \TLista{\bool}, \In apuestas:
\TLista{\TLista{\float}}, \In pagos: \TLista{\TLista{\float}}, \In eventos: \TLista{\TLista{\nat}}}{}
	\requiere{mismaLongitud(old(trayectorias),cooperan,apuestas,pagos,eventos)\land trayectoriasValidas(old(trayectorias))\\\land  pagosPositivos(pagos) \land apuestasValidas(apuestas) \land longitugApuestasPagos(apuestas,pagos) \land longitudSublistas(pagos) \\\land longitudSublistas(apuestas) \land longitudSublistas(eventos)}
	\asegura{(\forall i : \ent)(0 \leq i < |old(trayectorias)| \implicaLuego (trayectorias[i][0] = (old(trayectorias)[i])[0]) \yLuego (\forall j : \ent)(0 \leq i < |eventos[i]| \implicaLuego trayectorias[i][j+1]=calculoDeRecursosSegunCooperacion(trayectorias,pagos,apuestas,eventos,\\cooperan,i,j)))}
	\aux{calculoDeRecursosSegunCooperacion}{trayectorias,pagos,apuestas:\TLista{\TLista{\float}},eventos:\TLista{\TLista{\nat}},cooperan:\\\TLista{\bool},individuo:\ent,ronda:\ent}{\float}{if cooperan[individuo] = true then (fondoMonetario(trayectorias,pagos,apuestas,\\eventos,cooperan,ronda)/|cooperan|) else (calculoRecursos(trayectorias[individuo][ronda],\\pagos[individuo][eventos[individuo][ronda]],apuestas[individuo[eventos[individuo][ronda]]])+\\(fondoMonetario(trayectorias,pagos,apuestas,eventos,cooperan,ronda)/|cooperan|)) fi}
    \aux{calculoRecursos}{recurso,pago,apuesta:\float}{\float}{recurso*pago*apuesta}
    \aux{fondoMonetario}{trayectorias,pagos,apuestas:\TLista{\TLista{\float},eventos:\TLista{\TLista{\nat},cooperan:\TLista{\bool},ronda:\ent}}}{\float}{\sum\limits_{h=0}^{|cooperan|-1} if cooperan[h] = true then \\calculoRecursos(trayectorias[h][ronda],pagos[h][eventos[h][ronda]],apuestas[h][eventos[h][ronda]]) else 0 fi}
    \aux{sumaApuestas}{apuestas:\TLista{\TLista{\float}}, individuo: \ent}{\float}{\sum\limits_{h=0}^{|apuestas[individuo]|-1} apuestas[individuo][h]}
    \pred{mismaLongitud}{trayectorias,pagos,apuestas:\TLista{\TLista{\float}},eventos:\TLista{\TLista{\nat}},cooperan:\TLista{\bool}}{|trayectorias| > 0 \land |trayectorias|=|cooperan|=|apuestas|=|pagos|=|eventos|}
    \pred{trayectoriasValidas}{trayectorias:\TLista{\TLista{\float}}}{(\forall i: \ent)(0 \leq i < |trayectorias| \implicaLuego |trayectorias[i]| = 1 \yLuego (\forall x:\float)(x \in trayectorias[i] \implicaLuego x > 0))}
    \pred{pagosPositivos}{pagos:\TLista{\TLista{\float}}}{(\forall i,j:\ent)(0 \leq i < |pagos| \yLuego (0 \leq j < |pagos[i]| \implicaLuego pagos[i][j] > 0))}
    \pred{apuestasValidas}{apuestas:\TLista{\TLista{\float}}}{(\forall i,j:\ent)(0 \leq i < |apuestas| \implicaLuego sumaApuestas(apuestas,i) = 1 \yLuego (0 \leq j < |apuestas[i]| \implicaLuego 0 \leq apuestas[i][j] \leq 1)}
    \pred{longitudApuestasPagos}{apuestas,pagos:\TLista{\TLista{\float}}}{(\forall i:\ent)(0 \leq i < |apuestas| \implicaLuego |apuestas[i]| = |pagos[i]|)}
    \pred{longitudSublistas}{lista:\TLista{\TLista{T}}}{(\forall i:\ent)(0 \leq i < |lista|-1 \implicaLuego (|lista[i]| > 0 \yLuego |lista[i]| = |lista[i+1]|))}
\end{proc}



\subsection{Trayectoria Extraña Escalera}

\begin{proc}{trayectoriaExtrañaEscalera}{\In trayectoria: \TLista{\float}}{\bool}
	\requiere{|trayectoria| > 0}
	\asegura{(\forall u : \ent) \Biggl(  0 \leq i < |trayectoria| \land \biggl( \Bigl(  (1 \leq |trayectoria| \leq 2) \implies (res == True) \Bigr) \lor \Bigl(  hayUnicoMax(trayectoria) \implies  (res == True)\Bigr)   \biggr) \Biggr)}
	\pred{hayUnicoMax}{\In list: \TLista{\float}}{} 
\end{proc}

\subsection{Individuo Decide Si Cooperar O No}

\begin{proc}{individuoDecideSiCooperarONo}{\In individuo: \nat, \In recursos: \TLista{\float}, \Inout cooperan: \TLista{\bool}, \In apuestas: \TLista{\TLista{\float}}, \In pagos: \TLista{\float}, \In eventos: \TLista{\TLista{\nat}}}{}
	\requiere{expresionBooleana1}
	\asegura{expresionBooleana2}
	\aux{auxiliar1}{parametros}{tipoRes}{expresion}
	\pred{pred1}{parametros}{expresion} 
\end{proc}


\subsection{Individuo Actualiza Apuesta}

\begin{proc}{individuoActualizaApuesta}{\In individuo: \nat, \In recursos: \TLista{\float}, \In cooperan: \TLista{\bool}, \Inout apuestas: \TLista{\TLista{\float}}, \In pagos: \TLista{\TLista{\float}}, \In eventos: \TLista{\TLista{\nat}}}{}
	\requiere{expresionBooleana1}
	\asegura{expresionBooleana2}
	\aux{auxiliar1}{parametros}{tipoRes}{expresion}
	\pred{pred1}{parametros}{expresion} 
\end{proc}







\section{Demostracion de correctitud}

\begin{proc}{nombre}{\In paramIn : \nat, \Inout paramInout : \TLista{\ent}}{tipoRes}
	%    \modifica{parametro1, parametro2,..}
	\requiere{expresionBooleana1}
	\asegura{expresionBooleana2}
	\aux{auxiliar1}{parametros}{tipoRes}{expresion}
	\pred{pred1}{parametros}{expresion} 
\end{proc}

\aux{auxiliarSuelto}{parametros}{tipoRes}{expresion}
% \paraTodo{variable}{tipo}{expresion}
% \existe{variable}{tipo}{expresion}
% Pueden tener [unalinea] para que no se divida en varias lineas
\pred{predSuelto}{parametros}{\paraTodo[unalinea]{variable}{tipo}{algo \implicaLuego expresion}}
\pred{predSuelto}{parametros}{\existe[unalinea]{variable}{tipo}{algo \yLuego expresion}}



\end{document}
